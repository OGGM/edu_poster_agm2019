% -------------------------------------------------------------------
% - NAME:        poster.tex
% - AUTHOR:      Reto Stauffer
% - BASED ON:    Jakob Messners version of this theme
% - DATE:        2014-09-29
% -------------------------------------------------------------------
% - DESCRIPTION: This is a demo template for portrait beamer poster
%                in the UIBK design 2017.
% -------------------------------------------------------------------
\pdfminorversion=4 % Fixed a bug where rendered pdf's were not readable on windows/acroread
\documentclass[final]{beamer} 

\usepackage[utf8]{inputenc}
\usepackage[T1]{fontenc}

\usepackage{graphicx}
%\usepackage[orientation=portrait,size=a0,scale=1.30]{beamerposter}
%\usetheme[ncols=2]{uibkposter}
%% ------------------------------------------------------------------
%% Use the two lines above for portrait posters
%% ------------------------------------------------------------------
% Vorher: \usepackage[orientation=landscape,size=a0,scale=1.30]{beamerposter}
\usepackage[orientation=portrait,size=a0,scale=1.30]{beamerposter}
\usetheme[ncols=2,orangetheme]{uibkposter}
\usepackage{transparent}
\usepackage{multicol}
\usepackage{overpic}
\usepackage{wrapfig}
\usepackage{float}
\usepackage{caption}


%\usebackgroundtemplate%
%{%
%	\includegraphics[width=\paperwidth,height=\paperheight]{background4}%
%}

\headerimage{4}
%% ------------------------------------------------------------------
%% The theme offers four different header images based on the
%% corporate design of the university of innsbruck. Currently
%% 1, 2, 3 and 4 is allowed as input to \headerimage{...}. Default
%% or fallback is '1'.
%% ------------------------------------------------------------------

%% ------------------------------------------------------------------
%% The official corporate colors of the university are predefined and
%% can be used for e.g., highlighting something. Simply use
%% \color{uibkorange} or \begin{color}{uibkorange} ... \end{color}
%% Defined colors are:
%% - uibkorange, uibkblue, uibkgray, uibkgraym
%% Please note that there are two faculty colors (see definition above)
%% - uibkcol, uibkcoll
%% The frametitle color can be easily adjusted e.g., to black with
%% \setbeamercolor{titlelike}{fg=black}
%% ------------------------------------------------------------------

%\setbeamercolor{verbcolor}{fg=uibkorange}
%% ------------------------------------------------------------------
%% Setting a highlight color for verbatim output such as from
%% the commands \pkg, \email, \file, \dataset 
%% ------------------------------------------------------------------

%% Defines symbols for itemize that can be chosen individually (included by Zora)
\newcommand*{\smalllogo}[1]{%
	\raisebox{-.3\baselineskip}{%
		\includegraphics[
		height=\baselineskip,
		width=\baselineskip,
		keepaspectratio,
		]{#1}%
	}%
}
%% The title of your poster
% LOGO on the right side
\title{OGGM Edu \\ \vspace{1cm} A collaborative educational platform \\ about glaciers}

% LOGO on the left side
%\title{{\includegraphics[width=0.32\textwidth]{oggm_loupe}}\\\vspace*{-10cm} \hspace*{0.33\textwidth}OGGM Edu:\\\hspace*{0.33\textwidth}A collaborative educational platform\\\hspace*{0.33\textwidth}about glaciers}
%% If the subtitle is not set or empty no subtitle will be shown
\subtitle{http://edu.oggm.org}
%% Author(s) of the poster
\author{Zora L. Schirmeister (\textit{zora.schirmeister@uibk.ac.at}), Fabien Maussion} 

%% Enable numbered captions (figures, tables)
\setbeamertemplate{caption}[numbered]

%% Begin document
\begin{document}

\begin{frame}[fragile]

\vspace*{-14cm}\hspace*{0.625\textwidth}{\includegraphics[width=0.36\textwidth]{oggm_loupe_niedriger}}
\vspace*{3cm}

\centering
%% first box
		\begin{boxblock}{}
						{\large \color{uibkblue} An open source webpage where learning resources about glaciers and glacier modelling can be developed collaboratively.}
		\end{boxblock}

\vspace*{0.5cm}

% second row with two boxes
	\begin{minipage}[t]{36cm}
		\vspace{0pt}
		\begin{boxblock}{Aims}
				\begin{itemize}
						\item Interactive education \\$\rightarrow$ exercises for students and material for teachers
						\item Collaboration with anyone interested! \\$\rightarrow$ collective development
						\item Science related communication
						\item Open reproducible science
				\end{itemize}
		\end{boxblock}
	\end{minipage}
\hspace*{7.15cm}
%\hfill
	\begin{minipage}[t]{36cm}
		\vspace{0pt}
		\begin{boxblock}{Setup: Open source and cloud systems}
				\begin{itemize}
						\item OGGM (Open global glacier model)
						\item Open online repository for code (Github)
						\item Open programms (Jupyter notebook, JupyterLab)
						\item Open online programming environments (Binder)
						\item Open data (CRU, RGI...)
						\item Open license
				\end{itemize}
		\end{boxblock}
	\end{minipage}

%% first big block
		\begin{boxblock}{\hspace*{1cm}Content}
			\begin{itemize} \item[\smalllogo{lupe_rot.png}] \textbf{Interactive application} \end{itemize}
			\begin{multicols}{2}
						{\normalsize{
						With this app climate and geography of glaciers can be explored world wide. It es easy to handle, so that everyone interested is able to use it.\\
						\vspace*{1cm} 
						\textbf{Questions}, that can be answered:
				\begin{itemize}
						\item \textbf{Where} are the wettest glaciers located? And the driest?
						\item Is there a relationship between temperature and precipitation?
						\item \textbf{How much} glacier area is found on Greenland? In the European Alps? 
						\item What is the relationship between latitude and \textbf{glacier elevation}?
				\end{itemize}
						\vspace*{1cm} 
						\textbf{Selections} by:
				\begin{itemize}
						\item	region
						\item	elevation and latitude
						\item	annual precipitation
						\item	annual temperature
				\end{itemize}
				}}
			\columnbreak
				\begin{figure}
						\includegraphics[width=0.50\textwidth]{app_map} 
						\caption{World map in which glaciers can be selected in the app.}
				\end{figure}
			\end{multicols}
	\end{boxblock}

\vspace*{-1cm}

% begin columns (only right column, on left column is the background picture)
\begin{columns}
% big background picture on the left side
	\begin{picture}(0,0)
	%\put(140,-360){\includegraphics[width=0.4\textwidth]{gluehbirne}}
	\put(20,-550){\includegraphics[width=0.5\textwidth]{glacier_06}}
	%\put(50,-525){\includegraphics[width=0.3\textwidth]{ELA_04}}
	\end{picture}
	
	\hspace*{40.9cm}
	
% begin	right column with 3 boxes
\begin{rightcolumn}
		
		\hspace*{0pt}
		\vspace*{-1.3cm}
				
	\begin{boxblock}{}			
				\begin{itemize} \item[\smalllogo{globus.png}] \textbf{Interactive Notebooks} \end{itemize}	
						\normalsize{Interactive \textbf{modelling experiments} in jupyter notebook about glaciers for students. It is written in Python and based on OGGM (Open Global Glacier Model).
				\begin{multicols}{2}
						\textbf{Questions}, that are answered:
				\begin{itemize}
						\item \textbf{How} can a glacier be modeled?
						\item \textbf{Which} parameters describe ice flow?
						\item \textbf{What} is a surging glacier?
						\item \textbf{How} dies mass-balance influence the behaviour of glaciers?
				\end{itemize}
					%\columbreak
				\begin{figure}
						\includegraphics[width=0.35\textwidth]{notebooks_example_figure}
						\caption{Example code and output of a jupyter notebook cell.}
				\end{figure}
				\end{multicols}
			}
	\end{boxblock}
	
\hspace*{42cm}
\vspace*{-2.5cm}

	\begin{boxblock}{}
		\begin{minipage}[t]{0.45\textwidth}
				\begin{itemize}	\item[\smalllogo{gluehbirne.png}] {\textbf{Educational Graphics}} \end{itemize}
						Library for \textbf{educational material} like the graphics on the left and on the right.
		\end{minipage}
	\hfill
		\begin{minipage}[t]{0.35\textwidth}
						\vspace{0pt}
						\includegraphics[width=1.0\textwidth]{ELA_Bilder}
		\end{minipage}
	\end{boxblock}

\hspace*{42cm}
\vspace*{-1cm}

	\begin{boxblock}{Collaboration}
		\begin{minipage}[t]{0.9\textwidth}
				\begin{itemize}	\item[\smalllogo{kopf.png}] Everyone interested in this project is invited to participate, either by using it or better, by collaborating. Join us with all your ideas and suggestions for improvements!
				\end{itemize}
				\textbf{How?} The homepage is saved as a repository on \textbf{GitHub}. 
		\end{minipage}
		\hfill
		\begin{minipage}[t]{0.02\textwidth}
				\begin{figure}
						\includegraphics[width=\textwidth]{github_logo}
				\end{figure}
		\end{minipage}
		% \vspace{1em}
	\end{boxblock}
\end{rightcolumn}
\end{columns}

 %% References
\begin{footnotesize}
	
	\vspace{0.3cm}
	\begin{minipage}[t]{0.75\textwidth}
		\textbf{References:} \\
		%\bibliographystyle{ametsoc}
		%\bibliography{EMS}
		http://edu.oggm.org \\
		https://oggm.org
		\vspace{0.7cm}
		
		\textbf{Acknowledgements:} \\
		Thanks to all contributors of OGGM and OGGM-Edu.
		\\ Zora Schirmeister's work has been funded by the e-Learning department of the University of Innsbruck (Neue Medien Projekte - Call 18.03)
	\end{minipage}
	\hspace{6cm}
	\begin{minipage}[t]{0.1\textwidth}
		\begin{figure}
			\includegraphics[width=\textwidth]{qr-code}
		\end{figure}
	\end{minipage}
	\begin{minipage}[t]{0.05\textwidth}
		\begin{figure}
			\includegraphics[width=\textwidth]{license_ccby}
		\end{figure}
	\end{minipage}
\end{footnotesize}
\end{frame}
%%%%%%%%%%%%%%%%%%%%%%%%%%%%%%%%%%%%%%%%%%%%%%%%%%%%%%%%%%%%%%%%%%%%%%%
%%%%%%%%%%%%%%%%%%%%%%%%%%%%%%%%%%%%%%%%%%%%%%%%%%%%%%%%%%%%%%%%%%%%%%%
%%%%%%%%%%%%%%%%%%%%%%%%%%%%%%%%%%%%%%%%%%%%%%%%%%%%%%%%%%%%%%%%%%%%%%%
%\end{frame}

\end{document}
